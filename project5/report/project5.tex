\documentclass{article}
\usepackage[utf8]{inputenc}
\usepackage{amsmath}
\usepackage{xcolor}
\usepackage[top=2cm, bottom=2cm, left=2cm, right=2cm]{geometry}
\setlength\parindent{0pt}

\usepackage{listings}
%\usepackage{color}
\usepackage{graphicx}
\usepackage{float}
\usepackage{caption}

\usepackage{verbatim}
\let\oldv\verbatim
\let\oldendv\endverbatim

%\userpackage{minted}

\definecolor{dkgreen}{rgb}{0,0.6,0}
\definecolor{gray}{rgb}{0.5,0.5,0.5}
\definecolor{mauve}{rgb}{0.58,0,0.82}
\definecolor{light-gray}{gray}{0.95}


\lstset{frame=tb,
  language=Java,
  aboveskip=6mm,
  belowskip=6mm,
  showstringspaces=false,
  columns=flexible,
  basicstyle={\small\ttfamily},
  numbers=none,
  numberstyle=\tiny\color{gray},
  keywordstyle=\color{blue},
  commentstyle=\color{dkgreen},
  stringstyle=\color{mauve},
  breaklines=true,
  breakatwhitespace=true,
  tabsize=3,
  backgroundcolor=\color{light-gray},
  language=Matlab
}

%\usepackage{natbib} replaced by line below to make refernces work
\usepackage[square,sort,comma,numbers]{natbib}
\usepackage[nottoc,numbib]{tocbibind} %to get references in table of contants
\usepackage{graphicx}

\usepackage{bm}

\usepackage{hyperref}
\hypersetup{
	colorlinks,
	citecolor=black,
	filecolor=black,
	linkcolor=black,
	urlcolor=black
}

\usepackage{mdframed}
\usepackage{lipsum} % for creating dummy text
\mdfdefinestyle{MyFrame}{%
	linecolor=black,	
	backgroundcolor=gray!20!white,
	skipbelow = 8mm,
	skipabove = 8mm}

\usepackage{scrextend}

\usepackage{multimedia}
\usepackage{media9}

\usepackage{booktabs}
\usepackage{adjustbox}

\title{Fys4150\\Project 5\\ }
\author{Peter Killingstad and Karl Jacobsen\\
\\
\url{https://github.com/kaaja/fys4150}}

\begin{document}
	
\maketitle



\section{5a}

\subsection{Derivation of schemes with truncation errors}
All the schemes will be derived from Taylor series expansions, and the truncation error will be related to the remainder in the Taylor-series expansions. This remainded is the error we get when truncating the series by leaving out the remainder.\\

\subsubsection{Forward Euler}
For the time derivative, we expand $u(x, t + \Delta t)$ around $t$

\begin{subequations}
	\begin{align}
		u(x, t+ \Delta t)  = u(x,t) +  u_t(x,t) \Delta t + \mathcal{O}(\Delta t^2)\\
		\rightarrow u_t(x,t) = \frac{u(x, t+ \Delta t) - u(x,t)}{\Delta t} + \mathcal{O}(\Delta t)\label{eq:FeTime}
	\end{align}
\end{subequations}

The space derivative, which is a 2nd derivative, we derive by combining two Taylor series'

\begin{subequations}
	\begin{align}
		u(x + \Delta x,t) = u(x,t) + u_x(x,t)\Delta x + \frac{u_{xx}(x,t) \Delta x^2}{2} + \frac{u_{xxx}(x,t) \Delta x^3}{6} + \mathcal{O}(\Delta x^4)\label{eq:feSpace1}\\
		u(x - \Delta x,t) = u(x,t) - u_x(x,t)\Delta x + \frac{u_{xx}(x,t) \Delta x^2}{2} - \frac{u_{xxx}(x,t) \Delta x^3}{6} + \mathcal{O}(\Delta x^4)\label{eq:feSpace2}
	\end{align}
\end{subequations}

Now we add (\ref{eq:feSpace1}) and (\ref{eq:feSpace2}) and solve for $u_{xx}(x,t)$

\begin{subequations}
	\begin{align}
		\begin{split}
			\Big(u(x + \Delta x,t) + u(x - \Delta x,t) \Big) &= \Big(u(x,t) + u(x,t) \Big)\\ 
			&+ \Big(u_x(x,t)\Delta x + (- u_x(x,t)\Delta x) \Big)\\ 
			&+ \Big(\frac{u_{xx}(x,t) \Delta x^2}{2} +  \frac{u_{xx}(x,t) \Delta x^2}{2}\Big)\\ 
			&+ \Big(\frac{u_{xxx}(x,t) \Delta x^3}{6}  + (- \frac{u_{xxx}(x,t) \Delta x^3}{6}) \Big)\\ 
			&+ \Big(\mathcal{O}(\Delta x^4) + \mathcal{O}(\Delta x^4) \Big)
		\end{split}\\
		&= 2u(x,t) + u_{xx}(x,t) \Delta x^2 + \mathcal{O}(\Delta x^4)\\
		\rightarrow u_{xx}(x,t) &= \frac{u(x - \Delta x, t) - 2u(x,t) + u(x+ \Delta x, t)}{\Delta x^2} + \mathcal{O}(\Delta x^2)\label{eq:feSpace3}
	\end{align}
\end{subequations}

Combining (\ref{eq:FeTime}) and (\ref{eq:feSpace3}) we get the Forward Euler scheme

\begin{subequations}
	\begin{align}
		u_t(x,t) &= u_{xx}(x,t)\\
		\frac{u(x, t+ \Delta t) - u(x,t)}{\Delta t} + \mathcal{O}(\Delta t) &= 
		\frac{u(x - \Delta x, t) - 2u(x,t) + u(x+ \Delta x, t)}{\Delta x^2} + \mathcal{O}(\Delta x^2)\label{eq:fe}
	\end{align}
\end{subequations}

From (\ref{eq:fe}) we see that the scheme has a truncation error that goes like $\mathcal{O} (\Delta t)$ in time and $\mathcal{O}(\Delta x^2)$ in space.\\

We will now analyze the stability of the Forward Euler scheme (\ref{eq:fe}) by applying Neuman stability analyzis. From the analytical solution of the problem, we know that the particular solutions are on the form $u = e^{-(k \pi)^2 t}e^{i k \pi x}$, where $k$ is an integer greater than one. We observe that the solutions are stable in $t$, meaning that the solutions do not blow up as $t$ increaes. Based on the analytical particular solution, we make the numerical ansatz 

\begin{equation}\label{eq:neumanAnsatz}
	u = a_k^n e^{i k \pi x_j}
\end{equation}

For the numerical ansatz (\ref{eq:neumanAnsatz}) to reproduce the characteristics of the analytical particular solution, with stability in $t$, we observe that $|a_k^n| < 1$ in necessary. We now plug in the ansatz (\ref{eq:neumanAnsatz}) into the (\ref{eq:fe}) and derive an equation for $|a_k^n|$:

\begin{subequations}\label{eq:naumanFe0}
	\begin{align}
		\frac{u(x, t+ \Delta t) - u(x,t)}{\Delta t}  &= 
		\frac{u(x - \Delta x, t) - 2u(x,t) + u(x+ \Delta x, t)}{\Delta x^2} \\
		\frac{a_k^{n+1} e^{i k \pi (j+1) \Delta x} - a_k^n e^{i k \pi j \Delta x}}{\Delta t}  &= 
		\frac{a_k^{n} e^{i k \pi (j-1) \Delta x} - 2a_k^{n} e^{i k \pi j \Delta x} + a_k^{n} e^{i k \pi (j+1) \Delta x}}{\Delta x^2} \\
		a_k^{n} e^{i k \pi j \Delta x}\; \frac{a_k  -1}{\Delta t}  &= 
		a_k^{n} e^{i k \pi j \Delta x}\; \frac{ e^{-i k \pi \Delta x} - 2  +  e^{i k \pi  \Delta x}}{\Delta x^2} \\
		 \frac{a_k  -1}{\Delta t}  &= 
		 \frac{ e^{-i k \pi \Delta x} - 2  +  e^{i k \pi  \Delta x}}{\Delta x^2} \\
		 a_k &= 1 + \frac{\Delta t}{\Delta x^2} (e^{-i k \pi \Delta x} - 2  +  e^{i k \pi  \Delta x})\\
		 &= 1+ \frac{\Delta t}{\Delta x^2} \Big(2 \cos(k\pi\Delta x) - 2\Big)\\
		 &= 1+ 2\frac{\Delta t}{\Delta x^2} \Big( \cos(k\pi\Delta x) - 1\Big)\\
		 &= 1+ 2\frac{\Delta t}{\Delta x^2} \Big(- 2 \sin^2(\frac{k\pi\Delta x}{2}) \Big)\\
		 &=1 - 4\frac{\Delta t}{\Delta x^2} \sin^2(\frac{k\pi\Delta x}{2}) \\
		 |a_k| &= |1 - 4\frac{\Delta t}{\Delta x^2} \sin^2(\frac{k\pi\Delta x}{2})|\label{eq:neumanFe1}
	\end{align}
\end{subequations}

From (\ref{eq:neumanFe1}) we get
 
\begin{subequations}
	\begin{align}
		 &|a_k| < 1\; \text{if}\; ||1 - 4\frac{\Delta t}{\Delta x^2} \sin^2(\frac{k\pi\Delta x}{2})|| < 1 \\
		 &\rightarrow |1 - 4 \frac{\Delta t}{\Delta x^2}| < 1 \rightarrow |a_k| < 1\;\text{(Since $\sin^2(k \pi \Delta x/2)_{max}  = 1$)}\\
		 &\rightarrow 1 - 4 \frac{\Delta t}{\Delta x^2} > -1\\ 
		 &\rightarrow \frac{\Delta t}{\Delta x^2} < \frac{1}{2}\label{eq:neumanFe2}
	\end{align}
\end{subequations}

(\ref{eq:neumanFe2}) gives that the Forward Euler scheme is conditionally stable, and the condition that ensures stability.


\subsection{Backward Euler}
Here we will do the same as we did for Forward Euler above: Derive the scheme, including truncation errors, and analyze stability.\\

The only change compared to Forward Euler, is the time discretization, which now becomes

\begin{subequations}
	\begin{align}
	u(x, t- \Delta t)  = u(x,t) +  u_t(x,t) \Delta t - \mathcal{O}(\Delta t^2)\\
	\rightarrow u_t(x,t) = \frac{u(x, t) - u(x,t - \Delta t)}{\Delta t} + \mathcal{O}(\Delta t)\label{eq:beTime}
	\end{align}
\end{subequations}

The space discretization is the same as for Forward Euler, (\ref{eq:feSpace3}). Combining the space discretization (\ref{eq:feSpace3}) and (\ref{eq:beTime}) gives

\begin{subequations}
	\begin{align}
		\frac{u(x, t) - u(x,t - \Delta t)}{\Delta t} + \mathcal{O}(\Delta t) = \frac{u(x - \Delta x, t) - 2u(x,t) + u(x+ \Delta x, t)}{\Delta x^2} + \mathcal{O}(\Delta x^2)\label{eq:be1}
	\end{align}
\end{subequations}

We note that the truncation errors have the same asymptoptic behavior as for the Forward Euler scheme. \\

Now lets check the stability of the Backward Euler scheme. We apply the same method as we did for Forward Euler, and insert the ansatz (\ref{eq:neumanAnsatz}) into (\ref{eq:be1}) to get

\begin{subequations}
	\begin{align}
		\frac{u(x, t) - u(x,t - \Delta t)}{\Delta t}  &= \frac{u(x - \Delta x, t) - 2u(x,t) + u(x+ \Delta x, t)}{\Delta x^2} \\
		\frac{a_k^n e^{i k \pi j \Delta x} - a_k^{n-1} e^{i k \pi j \Delta x}}{\Delta t}  &= \frac{a_k^n e^{i k \pi (j -1)\Delta x} - 2a_k^n e^{i k \pi j \Delta x} + a_k^n e^{i k \pi (1 + j)\Delta x}}{\Delta x^2} \\
		a_k^n e^{i k \pi j \Delta x}\; \frac{1 - a_k^{-1}}{\Delta t}&=
		a_k^n e^{i k \pi j \Delta x}\; \frac{e^{-i k \pi \Delta x} - 2 + e^{i k \pi \Delta x}}{\Delta x^2}\\
		\frac{1 - a_k^{-1}}{\Delta t}&= \frac{e^{-i k \pi \Delta x} - 2 + e^{i k \pi \Delta x}}{\Delta x^2}\\
		a_k^{-1} &= 1 - \frac{\Delta t}{\Delta x^2} (e^{-i k \pi \Delta x} - 2 + e^{i k \pi \Delta x})\\
		a_k &= \frac{1}{1 - \frac{\Delta t}{\Delta x^2} (e^{-i k \pi \Delta x} - 2 + e^{i k \pi \Delta x})}\\
		&\stackrel{(\ref{eq:naumanFe0})}{=}  \frac{1}{1 + 4 \frac{\Delta t}{\Delta x^2} \sin^2(\frac{k\pi\Delta x}{2})}\\
		|a_k| &=  \left|\frac{1}{1 + 4 \frac{\Delta t}{\Delta x^2} \sin^2(\frac{k\pi\Delta x}{2})}\right| < 1.\label{eq:neumanBe}\\
	\end{align}
\end{subequations}

From (\ref{eq:neumanBe}) we see that, in contrast to the Forward Euler scheme, the Backward Euler scheme is unconditionally stable.\\


(\ref{eq:be1}) reveals another difference between the Backward Euler scheme and the Forward Euler scheme: (\ref{eq:be1}) is implicit in $u(x,t)$, meaning that we cannot solve (\ref{eq:be1}) directly for $u(x,t)$, as we did in the Forward Euler scheme. However, we can find $u(x,t)$ from (\ref{eq:be1}) by recognizing that (\ref{eq:be1}) can be rewritten as a linear system:

\begin{subequations}
	\begin{align}
		\frac{u(x, t) - u(x,t - \Delta t)}{\Delta t}  &= \frac{u(x - \Delta x, t) - 2u(x,t) + u(x+ \Delta x, t)}{\Delta x^2} \\
		\frac{u_i^n - u_i^{n-1}}{\Delta t}  &= \frac{u_{i-1}^n - 2u_i^n + u_{i+1}^n}{\Delta x^2} \\
		\frac{\Delta x^2}{\Delta t}(u_i^n - u_i^{n-1})  &=  u_{i-1}^n - 2u_i^n + u_{i+1}^n\\
		- \Big(u_{i-1}^n - (2 + \frac{\Delta x^2}{\Delta t}) u_i^n + u_{i+1}^n\Big)   &=  \frac{\Delta x^2}{\Delta t}u_i^{n-1}\\
		 \Big(-u_{i-1}^n + (2 + \frac{\Delta x^2}{\Delta t}) u_i^n - u_{i+1}^n\Big)   &=  \frac{\Delta x^2}{\Delta t}u_i^{n-1}\\
		\underbrace{\begin{bmatrix} (2 + \frac{\Delta x^2}{\Delta t}) & -1 & \cdots & 0 \\ -1 & (2 + \frac{\Delta x^2}{\Delta t}) & -1 & \vdots \\
			\vdots & &  \ddots & \vdots \\ 
			0 & \cdots & -1 & (2 + \frac{\Delta x^2}{\Delta t}) \end{bmatrix}}_{\mathbf{A}} 
		\underbrace{\begin{bmatrix} u_1^n\\ u_2^n \\ \vdots\\ u_N^n \end{bmatrix}}_{\mathbf{U}} &= 
		\underbrace{\frac{\Delta x^2}{\Delta t} \begin{bmatrix} u_0^{n-1}\\ u_1^{n-1} \\ \vdots \\ u_N^{n-1}\end{bmatrix}}_{\mathbf{\tilde{b}}}\label{eq:beLinSys}
	\end{align}
\end{subequations}

We see from (\ref{eq:beLinSys}) that solving Backward Euler corresponds to solving a linear system $A U = \tilde{b}$, where $A$ is a trdiagonal matrix. 
	

\end{document}

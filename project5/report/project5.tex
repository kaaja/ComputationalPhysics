\documentclass{article}
\usepackage[utf8]{inputenc}
\usepackage{amsmath}
\usepackage{xcolor}
\usepackage[top=2cm, bottom=2cm, left=2cm, right=2cm]{geometry}
\setlength\parindent{0pt}

\usepackage{listings}
%\usepackage{color}
\usepackage{graphicx}
\usepackage{float}
\usepackage{caption}

\usepackage{verbatim}
\let\oldv\verbatim
\let\oldendv\endverbatim

%\userpackage{minted}

\definecolor{dkgreen}{rgb}{0,0.6,0}
\definecolor{gray}{rgb}{0.5,0.5,0.5}
\definecolor{mauve}{rgb}{0.58,0,0.82}
\definecolor{light-gray}{gray}{0.95}


\lstset{frame=tb,
  language=Java,
  aboveskip=6mm,
  belowskip=6mm,
  showstringspaces=false,
  columns=flexible,
  basicstyle={\small\ttfamily},
  numbers=none,
  numberstyle=\tiny\color{gray},
  keywordstyle=\color{blue},
  commentstyle=\color{dkgreen},
  stringstyle=\color{mauve},
  breaklines=true,
  breakatwhitespace=true,
  tabsize=3,
  backgroundcolor=\color{light-gray},
  language=Matlab
}

%\usepackage{natbib} replaced by line below to make refernces work
\usepackage[square,sort,comma,numbers]{natbib}
\usepackage[nottoc,numbib]{tocbibind} %to get references in table of contants
\usepackage{graphicx}

\usepackage{bm}

\usepackage{hyperref}
\hypersetup{
	colorlinks,
	citecolor=black,
	filecolor=black,
	linkcolor=black,
	urlcolor=black
}

\usepackage{mdframed}
\usepackage{lipsum} % for creating dummy text
\mdfdefinestyle{MyFrame}{%
	linecolor=black,	
	backgroundcolor=gray!20!white,
	skipbelow = 8mm,
	skipabove = 8mm}

\usepackage{scrextend}

\usepackage{multimedia}
\usepackage{media9}

\usepackage{booktabs}
\usepackage{adjustbox}

\title{Fys4150\\Project 5\\ }
\author{Peter Killingstad and Karl Jacobsen\\
\\
\url{https://github.com/kaaja/fys4150}}

\begin{document}
	
\maketitle



\section{5a}

\subsection{Derivation of schemes with truncation errors}
All the schemes will be derived from Taylor series expansions, and the truncation error will be related to the remainder in the Taylor-series expansions. This remainded is the error we get when truncating the series by leaving out the remainder.\\

\subsubsection{Forward Euler}
For the time derivative, we expand $u(x, t + \Delta t)$ around $t$

\begin{subequations}
	\begin{align}
		u(x, t+ \Delta t)  = u(x,t) +  u_t(x,t) \Delta t + \mathcal{O}(\Delta t^2)\\
		\rightarrow u_t = \frac{u(x, t+ \Delta t) - u(x,t)}{\Delta t} + \mathcal{O}(\Delta t)\label{eq:FeTime}
	\end{align}
\end{subequations}

The space derivative, which is a 2nd derivative, we derive by combining two Taylor series'

\begin{subequations}
	\begin{align}
		u(x + \Delta x,t) = u(x,t) + u_x(x,t)\Delta x + \frac{u_{xx}(x,t) \Delta x^2}{2} + \frac{u_{xxx}(x,t) \Delta x^3}{6} + \mathcal{O}(\Delta x^4)\label{eq:feSpace1}\\
		u(x - \Delta x,t) = u(x,t) - u_x(x,t)\Delta x + \frac{u_{xx}(x,t) \Delta x^2}{2} - \frac{u_{xxx}(x,t) \Delta x^3}{6} + \mathcal{O}(\Delta x^4)\label{eq:feSpace2}
	\end{align}
\end{subequations}

Now we add (\ref{eq:feSpace1}) and (\ref{eq:feSpace2}) and solve for $u_{xx}(x,t)$

\begin{subequations}
	\begin{align}
		\begin{split}
			\Big(u(x + \Delta x,t) + u(x - \Delta x,t) \Big) &= \Big(u(x,t) + u(x,t) \Big)\\ 
			&+ \Big(u_x(x,t)\Delta x + (- u_x(x,t)\Delta x) \Big)\\ 
			&+ \Big(\frac{u_{xx}(x,t) \Delta x^2}{2} +  \frac{u_{xx}(x,t) \Delta x^2}{2}\Big)\\ 
			&+ \Big(\frac{u_{xxx}(x,t) \Delta x^3}{6}  + (- \frac{u_{xxx}(x,t) \Delta x^3}{6}) \Big)\\ 
			&+ \Big(\mathcal{O}(\Delta x^4) + \mathcal{O}(\Delta x^4) \Big)
		\end{split}\\
		&= 2u(x,t) + u_{xx}(x,t) \Delta x^2 + \mathcal{O}(\Delta x^4)\\
		\rightarrow u_{xx}(x,t) &= \frac{u(x - \Delta x, t) - 2u(x,t) + u(x+ \Delta x, t)}{\Delta x^2} + \mathcal{O}(\Delta x^2)\label{eq:feSpace3}
	\end{align}
\end{subequations}

Combining (\ref{eq:FeTime}) and (\ref{eq:feSpace3}) we get the Forward Euler scheme

\begin{subequations}
	\begin{align}
		u_t(x,t) &= u_{xx}(x,t)\\
		\frac{u(x, t+ \Delta t) - u(x,t)}{\Delta t} + \mathcal{O}(\Delta t) &= 
		\frac{u(x - \Delta x, t) - 2u(x,t) + u(x+ \Delta x, t)}{\Delta x^2} + \mathcal{O}(\Delta x^2)\label{eq:fe}
	\end{align}
\end{subequations}

From (\ref{eq:fe}) we see that the scheme has a truncation error that goes like $\mathcal{O} (\Delta t)$ in time and $\mathcal{O}(\Delta x^2)$ in space.\\

We will now analyze the stability of the Forward Euler scheme (\ref{eq:fe}) by applying Neuman stability analyzis. From the analytical solution of the problem, we know that the particular solutions are on the form $u = e^{-(k \pi)^2 t}e^{i k \pi x}$, where $k$ is an integer greater than one. We observe that the solutions are stable in $t$, meaning that the solutions do not blow up as $t$ increaes. Based on the analytical particular solution, we make the numerical ansatz $u = a_k^n e^{i k \pi x_j}$. For the numerical ansatz to reproduce the characteristics of the analytical particular solution, with stability in $t$, we observe that $|a_k^n| < 1$ in necessary.





\end{document}

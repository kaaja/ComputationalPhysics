\documentclass{article}
\usepackage[utf8]{inputenc}
\usepackage{amsmath}
\usepackage{xcolor}
\usepackage[top=2cm, bottom=2cm, left=2cm, right=2cm]{geometry}
\setlength\parindent{0pt}

\usepackage{listings}
%\usepackage{color}
\usepackage{graphicx}
\usepackage{float}
\usepackage{caption}

\usepackage{verbatim}
\let\oldv\verbatim
\let\oldendv\endverbatim

%\userpackage{minted}

\definecolor{dkgreen}{rgb}{0,0.6,0}
\definecolor{gray}{rgb}{0.5,0.5,0.5}
\definecolor{mauve}{rgb}{0.58,0,0.82}
\definecolor{light-gray}{gray}{0.95}


\lstset{frame=tb,
  language=Java,
  aboveskip=6mm,
  belowskip=6mm,
  showstringspaces=false,
  columns=flexible,
  basicstyle={\small\ttfamily},
  numbers=none,
  numberstyle=\tiny\color{gray},
  keywordstyle=\color{blue},
  commentstyle=\color{dkgreen},
  stringstyle=\color{mauve},
  breaklines=true,
  breakatwhitespace=true,
  tabsize=3,
  backgroundcolor=\color{light-gray},
  language=Matlab
}

%\usepackage{natbib} replaced by line below to make refernces work
\usepackage[square,sort,comma,numbers]{natbib}
\usepackage[nottoc,numbib]{tocbibind} %to get references in table of contants
\usepackage{graphicx}

\usepackage{bm}

\usepackage{hyperref}
\hypersetup{
	colorlinks,
	citecolor=black,
	filecolor=black,
	linkcolor=black,
	urlcolor=black
}

\usepackage{mdframed}
\usepackage{lipsum} % for creating dummy text
\mdfdefinestyle{MyFrame}{%
	linecolor=black,	
	backgroundcolor=gray!20!white,
	skipbelow = 8mm,
	skipabove = 8mm}

\usepackage{scrextend}

\usepackage{multimedia}
\usepackage{media9}

\usepackage{booktabs}
\usepackage{adjustbox}

\title{Fys4150\\Project 5 Analytical solution\\ }
\author{Peter Killingstad and Karl Jacobsen}
\begin{document}
\maketitle
\section*{Analytical solution 1D}
The one-dimensional problem is written as 

\begin{subequations}
\begin{eqnarray}
	\frac{\partial^2 u(x,t)}{\partial x^2} &=& \frac{\partial u(x,t)}{\partial t} \textit{, } t> 0 \textit{, } x \in [0,L] \\ \nonumber
\\
u(x,0) &=& 0 \textit{, } 0 <x < L \textit{,} \\
u(0,t) &=& 0 \textit{, } t>0 \\
u(L,t) &=& 1 \textit{, } t>0
\end{eqnarray}
\end{subequations}
With initial conditions at $t=0$ and $L=1$ is the length of the $x$-region of interest.
\\
\\
The problem consists of non-homogeneous boundary conditions, and it is not trivial to find a closed form solution. The problem can be seen as a physical problem such as a temperature gradient in a rod or flow between two infinite flat plates, where the fluid is initially at rest and the plate at $x=1$ is given a sudden movement. From physical observations we know that as time goes, i.e. when $t\rightarrow \infty$,  the problem coinsides whith its surroundings and becomes steady.\\
\\ It is reasonable to introduce a solution consisting of the sum of two parts, a steady-state solution, and a transient solution which is the part that depends on the initial conditions.\\ \\
We end up with the solution of the problem beeing defined as
\begin{equation}
u(x,t) = U(x) + V(x,t)
\end{equation}
where $U(x)$ is the steady-state solution and $V(x,t)$ is the transient solution.\\ 
\\
Since the steady-state problem is not changing in time, i.e. $\partial_ t = 0$, it is written in a compact form as:
\begin{subequations}
\begin{eqnarray}
\label{eqn:steadyStateSODE}
U_{xx} &=& 0 \textit{ , } x \in [0,1]\\ \nonumber
\\
U(0) &=& 0 \textit{  } \\
U(L) &=& 1
\end{eqnarray}
\end{subequations}
For the transient part of the problem we have the problem written in compact form as:
\begin{subequations}
\begin{eqnarray}
\label{eqn:transientODE}
V_t &=& V_{xx} \textit{ , } t>0 \textit{ , } x \in [0,1] \\ \nonumber
\\
\label{eqn:transientIC}
V(x,0) &=& -U_s(x) \textit{ , } 0<x<1 \\
V(0,t) &=& 0 \textit{ , } t>0 \\
V(1,t) &=& 0 \textit{ , } t>0
\end{eqnarray}
\end{subequations}
\\
\\
The steady state solution is solved by integrating \ref{eqn:steadyStateSODE} twice
\begin{eqnarray}
\nonumber
\int \frac{\partial^2 u(x,t)}{\partial x^2} &=& A \\ \nonumber
\int \frac{\partial u(x,t)}{\partial x} &=& \int A \\ \nonumber
\end{eqnarray}
and we end up with the general solution $U(x) = Ax + B$. Applying the boundary conditions at $L=0$ and $L=1$ we get that 
\begin{equation}
U(x) = x
\label{eqn:UsteadyState1D}
\end{equation}
The transient problem \ref{eqn:transientODE} has homogeneous boundary conditions, and we can solve it by seperation of variables. We start by making the anzats that $V(x,t) = T(t)X(x)$ so that we can rewrite \ref{eqn:transientODE} as:
\begin{equation}
XT'=X''T 
\end{equation}
Reordering the equation we get the following:
\begin{equation}
\frac{T'}{T} = \frac{X''}{X} = k,
\end{equation}
where $k$ is an unknown, possible complex constant. The choosing of $k$ can be explained by the following arguments. We are looking for a solution that is eventually going to be steady. With a positive $k$ we would never get a steady state, as the factor $T(t)$ does not go to zero as $t \rightarrow \infty$. Choosing $k$ as zero would give us the steady-solution which we already have. We are the then left with choosing $k$ as a negative constant. With $k=-\lambda^2$ we ensure thath k is negative.\\
\\
The term can be solved seperately as we get that the left hand side is only depentent on $t$ and the right hand side is only depentent on $x$. They are also independent on each other.\\
\\
For the $t$ dependent part we get
\begin{eqnarray}
\frac{T'}{T} = -\lambda^2 \\ 
T' = -\lambda^2 T
\end{eqnarray}
The equation can be solved by using the anzats $T = e^{\alpha t}$. We get the following expression
\begin{subequations}
\begin{eqnarray}
\alpha e^{\alpha t} + \lambda^2 e^{\alpha t} = 0 \\ 
\alpha +\lambda^2 = 0 \\ 
\Rightarrow \alpha = -\lambda^2
\end{eqnarray}
\end{subequations}
The $t$ dependent term is then given by
\begin{equation}
T = Ae^{-\lambda^2 t}
\end{equation}
From this epression we see that the $\lambda$ must be real. Otherwise the term would grow whitout bounds over time, which is not in accordance with the behaviour we are looking for. 
\\
\\
For the $x$ dependent term we get
\begin{subequations}
\begin{eqnarray}
X'' = -\lambda^2 X \\
\Rightarrow X'' + \lambda^2 X = 0 
\end{eqnarray}
\end{subequations}

Again we use an anzats, that $X=e^{\alpha x}$:
\begin{subequations}
\begin{eqnarray}
\alpha^2 e^{\alpha x} + \lambda^2 e^{\alpha x} &=& 0 \\ 
\Rightarrow \alpha^2 &=& -\lambda^2 \\ 
\Rightarrow \alpha &=& \pm \sqrt{-\lambda^2}
\end{eqnarray}
\end{subequations}
By using that $\lambda$ must be a real number, the $x$ dependent term has the general solution 
\begin{equation}
X = Be^{-\lambda x} + Ce^{\lambda x}
\label{eqn:xDependenttransientTerm1D}
\end{equation}
We are looking for trigonometric solution to the term, so we rewrite the expression above as:
\begin{equation}
X(x) = Bcos(\lambda x) + Csin(\lambda x) 
\end{equation}
Applying the boundary conditions we get that
\begin{eqnarray}
X(0) = Bcos(0) + Csin(0) = 0 \\ 
X(1) = Bcos(\lambda) + Csin(\lambda) = 0
\end{eqnarray}
From the first boundary condition wher $L=0$ we must have that $B=0$. We are lookong for non-trivial solutions of the problem, so for $L=1$ we must find for which values the term $Csin(\lambda)=0$. We know that\\ $sin(n \pi) =0 \textit{ for } n = 1,2, \ldots $, which lets us determine that $\lambda = n \pi$.
\\
\\
We have then infinite solutions of the transient part on the form:
\begin{subequations}
\begin{eqnarray}
V_n(x,t) = A_ne^{-(n\pi)^2 t}C_nsin(n \pi x)\\
\Rightarrow a_ne^{-(n\pi)^2 t}sin(n \pi x) \textit{ for } n = 1,2, \ldots 
\end{eqnarray}
\end{subequations}
\\
Where ${V_n(x,t)}$ is the family of particular solutions we are looking for. Our desired solution will be a certain sum of this family of particular solutions i.e. fundamental solutions:
\begin{equation}
\sum_{n=1}^{\infty} a_ne^{-(n\pi)^2 t}sin(n \pi x)
\label{eqn:FundamentalSolutionODE}
\end{equation}
We must add the sum above in a way that the coefficients $a_n$ satisfies the the initial condition of the transient solution (\ref{eqn:transientIC}). This is done by setting 
\begin{equation}
-U(x) = \sum_{n=1}^{\infty} a_nsin(n \pi x) 
\end{equation}
To obtain an expression for the coefficients $a_n$ we use that the functions $sin(n\pi x)$ is orthogonal to each other in the sense that 
\begin{equation}
\int_0^1 sin(n\pi x) sin(m\pi x) dx = 
	\begin{cases} 0 & \quad \text{if } m \neq n\\
    1/2 & \quad \text{if } m = n 
    \end{cases}
    \label{eqn:orhogonalSine}
\end{equation}
Using the above statement and multiplying the particular solutions   \ref{eqn:FundamentalSolutionODE} by $\int_0^1sin(m\pi x)$ we get the expression
\begin{eqnarray}
\nonumber
- \int_0^1 U(x)sin(m\pi x) &=& a_n \int_0^1 sin(m\pi x)sin(n\pi x) = a_m\int_0^1 sin^2(m\pi x)\\ \nonumber
&\Rightarrow& - \int_0^1 U(x)sin(m\pi x) =\frac{a_m}{2}\\ 
&\Rightarrow& a_m = - 2\int_0^1 U(x)sin(m\pi x) 
\end{eqnarray}
solving for $a_n$ yields 
\begin{eqnarray}
\nonumber
a_n = -2 \int_0^1 x sin(m\pi x) &=& \Big[\frac{2}{(m\pi)^2} sin(m\pi x) + \frac{2x}{m\pi}cos(m\pi x)\Big]_0^1 \\ 
\label{eqn:fourierCoefficients}
&\Rightarrow & a_n = \frac{2}{m\pi}(-1)^m
\end{eqnarray}
\newline

Putting the coefficient from \ref{eqn:fourierCoefficients} into the fundamental solutions \ref{eqn:FundamentalSolutionODE} we end up with the following solution for the transient problem:
\begin{equation}
V(x,t) = 2\sum_{n=1}^{\infty} \frac{(-1)^k}{n\pi} e^{-(n\pi)^2 t}sin(n\pi x)
\label{eqn:transientSolution1D}
\end{equation}
Now we get an expression for the whole problem by putting combining the steady-state solution (\ref{eqn:UsteadyState1D}) and the transient solution (\ref{eqn:transientSolution1D}) such that we obtain 
\begin{equation}
u(x,t) = U(x) + V(x,t) = x + 2\sum_{n=1}^{\infty} \frac{(-1)^k}{n\pi}
sin(n\pi x)e^{-(n\pi)^2 t}
\label{eqn:solution1D}
\end{equation}

\section*{Analytical solution 2D}
In the two-dimensional case the differential equation becomes
\begin{subequations}
\begin{eqnarray}
\frac{\partial^2 u(x,y,t)}{\partial x^2} + \frac{\partial^2 u(x,y,t)}{\partial y^2} &=&  \frac{\partial u(x,y,t)}{\partial t} \textit{ , } t>0 \textit{ , } x,y \in [0,1] \\ \nonumber \\
u(x,y,0) &=& 0 \textit{  } \\ 
u(0,y,t) &=& 0 \textit{ , } t> 0 \\
u(1,y,t) &=& 0 \textit{ , } t> 0 \\
u(x,0,t) &=& 0 \textit{ , } t> 0  \\
u(x,1,t) &=& 1 \textit{ , } t> 0 
\end{eqnarray}
\end{subequations}
The boundary conditions is extended in such a way that the physical problem can be that of a flow whitin a box with infinite plates in the streamwise direction ($z$), where the fluid is initially at rest and the plate at $y=1$ is given a sudden movement. So we get "no slip" boundary conditions i.e. the fluid has zero movement on all boundaries except at $y=1$.
\\
\\
To obtain a closed form solution for the two-dimensional we use the same argument as for the one-dimensional proplem i.e. we split the solution into two parts, consisting of a steady-state solution and a transient solution.\\
\\
We end up with a solution defined as:
\begin{equation}
u(x,y,t) = U(x,y) + V(x,y,t)
\end{equation}
where $U(x,y)$ is the steady-state solution and $V/x,y,t)$ is the transient solution.\\
\\
The steady-solution is then written in a compact form as:
\begin{subequations}
\begin{eqnarray}
U_{xx} + U_{yy} &=& 0 \textit{ , } x,y \in [0,1]\\ \nonumber
\\
U(0,y) &=& 0 \textit{  } \\
U(1,y) &=& 0 \textit{  } \\
U(x,0) &=& 0 \textit{  } \\
U(x,1) &=& 1 \textit{  } 
\end{eqnarray}
\end{subequations}
For the transient problem we have the propblem defined as:
\begin{subequations}
\begin{eqnarray}
\label{eqn:transientPDE}
V_t &=& V_{xx} \textit{ , } t>0 \textit{ , } x \in [0,1] \\ \nonumber
\\
\label{eqn:transientICPDE}
V(x,0) &=& -U(x,y) \textit{ , } x,y \in [0,1] \\
V(0,y,t) &=& 0 \textit{ , } t>0 \\
V(1,y,t) &=& 0 \textit{ , } t>0 \\
V(x,0,t) &=& 0 \textit{ , } t>0 \\
V(x,1,t) &=& 0 \textit{ , } t>0 
\end{eqnarray}
\end{subequations}
\\\\
For the steady-state problem we have a 2D Laplacian equation, which we can solve by seperation of variables. We use the anzats that $U(x,y) = X(x)Y(y)$, so that we end up with:
\begin{eqnarray}
\nonumber
YX'' + XY''=0 \\ \nonumber
\frac{X''}{X} = - \frac{Y''}{Y} = -\beta^2
\end{eqnarray}
Due to the independence of the terms on each side of the equation, we can solve the equations separately, with each equation beeing
\begin{equation}
\frac{X''}{X} = -\beta^2 \nonumber
\end{equation}
\begin{equation}
\frac{Y''}{Y} = \beta^2 \nonumber
\end{equation}
The sign on the right hand side in both of the above equations es determined by the fact that in the $x$ dependent term we have homogeneous boundaries, while in the $y$ dependent term we have non-homogeneous. The $x$ independent term must be on a form that allows it to be homogeneous on the boundaries, while the $y$ term must be on a form that allows it to be 0 at $y=0$ and 1 at $y=1$. 

For the $x$ dependent term we get
\begin{eqnarray}
\nonumber
X'' = -\beta^2 X \\ \nonumber
\Rightarrow X'' + \beta^2 X = 0 
\end{eqnarray}
which has a similar solution as that of the transient $x$ dependent term in the one-dimensional case (\ref{eqn:xDependenttransientTerm1D}).We are again looking for a trigonometric form
\begin{equation}
X(x) = Acos(\beta x) + Bsin(\beta x) \nonumber
\end{equation}
Applying the boundary conditions we end up with the term
\begin{equation}
X_n(x) = B_nsin(n\pi x)
\label{eqn:xDependenSteadyState2D}
\end{equation}
were ${X_n(x)}$ is the family of particular solutions.\\
\\

The $y$ dependent term can be rewritten as 
\begin{eqnarray}
\nonumber
y'' = \beta^2 Y \\ \nonumber
\Rightarrow Y'' - \beta^2 Y = 0 
\end{eqnarray}
We are again using an anzats so that $Y = e^{\gamma y}$. This leaves us with the term
\begin{eqnarray}
\nonumber
\gamma^2 e^{\gamma y} - \beta^2 e^{\gamma y} &=& 0 \\ \nonumber
\Rightarrow \gamma^2 &=& \beta^2 \\ \nonumber
\Rightarrow \gamma &=& \pm \beta
\end{eqnarray}
and we end up with the following expression
\begin{eqnarray}
Y_n(y) = C_ne^{-n\pi y} + D_ne^{n\pi y} \\ 
Y_n(y) = C_ncosh(n\pi y) + D_nsinh(n\pi y)
\end{eqnarray}
where {$Y_n(y)$} is the family of particular solutions.
\\
\\
The fundamental solutions of the steady-state is the given by
\begin{equation}
U(x,y) = \sum_{n=1}^{\infty} X(x)Y(y) = \sum_{n=1}^{\infty} B_nsin(n\pi x)\Big(C_ncosh(n\pi y) + D_nsinh(n\pi y)\Big)
\end{equation}
Applying the the boundary of $U(x,0)$ to the equation above we get that $C_n$ must be zero. We are then left with the expression
\begin{equation}
 U(x,y) = \sum_{n=1}^{\infty} c_n sin(n\pi x)sinh(n\pi y)
\end{equation}
Applying the last boundary $U(x,1) = 1$ we must have that
\begin{equation}
1 = \sum_{n=1}^{\infty} c_n sin(n\pi x)sinh(n\pi)
\end{equation}
setting $d_n = c_nsinh(n\pi)$ we get that 
\begin{equation}
1 = \sum_{n=1}^{\infty} d_n sin(n\pi x)
\end{equation}
\\
and using again that the functions $sin(n\pi x)$ is orthogonal (\ref{eqn:orhogonalSine}), as in the 1D case, we get that 
\begin{subequations}
\begin{eqnarray}
dn = 2\int_0^1 sin(m\pi x) dx = \frac{2}{m\pi}(1-(-1)^m\\
\Rightarrow c_n = \frac{2}{m\pi sinh(m\pi)}(1-(-1)^m
\end{eqnarray}
\end{subequations}

This in turn lets us write the steady-state solution as 
\begin{subequations}
\begin{eqnarray}
U(x,y) = \frac{2}{\pi} \sum_{m=1}^{\infty} \frac{sin(m\pi x)sinh(m\pi y)}{\pi sinh(m\pi)}(1-(-1)^m \\ 
U(x,y) = \frac{4}{\pi} \sum_{m}^{\infty} \frac{sin((2m-1)\pi x)sinh((2m-1)\pi y)}{\pi sinh((2m-1)\pi)}
\end{eqnarray}
\end{subequations}




















\end{document}
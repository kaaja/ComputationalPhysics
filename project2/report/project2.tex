\documentclass{article}
\usepackage[utf8]{inputenc}
\usepackage{amsmath}
\usepackage{xcolor}
\usepackage[top=2cm, bottom=2cm, left=2cm, right=2cm]{geometry}
\setlength\parindent{0pt}

\usepackage{listings}
%\usepackage{color}
\usepackage{graphicx}
\usepackage{float}
\usepackage{caption}

\usepackage{verbatim}
\let\oldv\verbatim
\let\oldendv\endverbatim

%\userpackage{minted}

\definecolor{dkgreen}{rgb}{0,0.6,0}
\definecolor{gray}{rgb}{0.5,0.5,0.5}
\definecolor{mauve}{rgb}{0.58,0,0.82}
\definecolor{light-gray}{gray}{0.95}


\lstset{frame=tb,
  language=Java,
  aboveskip=6mm,
  belowskip=6mm,
  showstringspaces=false,
  columns=flexible,
  basicstyle={\small\ttfamily},
  numbers=none,
  numberstyle=\tiny\color{gray},
  keywordstyle=\color{blue},
  commentstyle=\color{dkgreen},
  stringstyle=\color{mauve},
  breaklines=true,
  breakatwhitespace=true,
  tabsize=3,
  backgroundcolor=\color{light-gray},
  language=Matlab
}

%\usepackage{natbib} replaced by line below to make refernces work
\usepackage[square,sort,comma,numbers]{natbib}
\usepackage[nottoc,numbib]{tocbibind} %to get references in table of contants
\usepackage{graphicx}

\usepackage{bm}

\usepackage{hyperref}
\hypersetup{
	colorlinks,
	citecolor=black,
	filecolor=black,
	linkcolor=black,
	urlcolor=black
}

\usepackage{mdframed}
\usepackage{lipsum} % for creating dummy text
\mdfdefinestyle{MyFrame}{%
	linecolor=black,	
	backgroundcolor=gray!20!white,
	skipbelow = 8mm,
	skipabove = 8mm}

\usepackage{scrextend}

\title{Fys4150\\Project 2\\ }
\author{Peter Killingstad and Karl Jacobsen\\
\\
\url{https://github.com/kaaja/fys4150}}


\begin{document}
	
\maketitle

\section*{Abstract}


\section{Introduction}


\section{Theory}




\section{Results}

\subsection{Algorithm efficiency}

\begin{minipage}{.49\textwidth} %\noindent
	\begin{figure}[H]
		\centering
		\includegraphics[width=0.99\textwidth]{/home/karl/doc/subj/att/fys4150/project2/resultsKeep/oneElectronArmadilloTimeDimensions.pdf}
		\caption{Algorithm times divided by Armadillo time}
		\label{1}
	\end{figure}
\end{minipage}\hfill
\begin{minipage}{.49\textwidth}
	\begin{figure}[H]
		\centering
		\includegraphics[width=0.99\textwidth]{/home/karl/doc/subj/att/fys4150/project2/resultsKeep/oneElectronComparisonLogTimeDimensions.pdf}
		\caption{log Algorithm times}
		\label{1}
	\end{figure}
\end{minipage}\hfill

\begin{minipage}{.49\textwidth} %\noindent
	\begin{figure}[H]
		\centering
		\includegraphics[width=0.99\textwidth]{/home/karl/doc/subj/att/fys4150/project2/resultsKeep/TwoElectronCoulombOmega50RhoMaxComparison.pdf}
		\caption{Comparison rhoMax. Minimum eigenvalue. Coulomb}
		\label{1}
	\end{figure}
\end{minipage}\hfill
\begin{minipage}{.49\textwidth}
	\begin{figure}[H]
		\centering
		\includegraphics[width=0.99\textwidth]{/home/karl/doc/subj/att/fys4150/project2/resultsKeep/TwoElectronCoulombOmega10RhoMaxComparison.pdf}
		\caption{Comparison rhoMax. Minimum eigenvalue. Coulomb}
		\label{1}
	\end{figure}
\end{minipage}\hfill
\begin{minipage}{.49\textwidth}
	\begin{figure}[H]
		\centering
		\includegraphics[width=0.99\textwidth]{/home/karl/doc/subj/att/fys4150/project2/resultsKeep/TwoElectronCoulombOmega05RhoMaxComparison.pdf}
		\caption{Comparison rhoMax. Minimum eigenvalue. Coulomb}
		\label{1}
	\end{figure}
\end{minipage}\hfill
\begin{minipage}{.49\textwidth}
	\begin{figure}[H]
		\centering
		\includegraphics[width=0.99\textwidth]{/home/karl/doc/subj/att/fys4150/project2/resultsKeep/TwoElectronCoulombOmega001RhoMaxComparison.pdf}
		\caption{Comparison rhoMax. Minimum eigenvalue. Coulomb}
		\label{1}
	\end{figure}
\end{minipage}\hfill

\subsection{Sturm bisection}

\begin{minipage}{.49\textwidth} %\noindent
	\begin{figure}[H]
		\centering
		\includegraphics[width=0.99\textwidth]{/home/karl/doc/subj/att/fys4150/project2/resultsKeep/oneElectronOmega10RhoMaxComparisonRevised.pdf}
		\caption{Minimum eigenvalue. One electron Sturm Bisection.}
		\label{1}
	\end{figure}
\end{minipage}\hfill
\begin{minipage}{.49\textwidth}
	\begin{figure}[H]
		\centering
		\includegraphics[width=0.99\textwidth]{/home/karl/doc/subj/att/fys4150/project2/resultsKeep/oneElectronRelativeErrorEigenvaluesRevised.pdf}
		\caption{Maximum relative error of three smallest eigenvalues. One electron. Sturm Bisection}
		\label{1}
	\end{figure}
\end{minipage}\hfill


\subsection{Comparison with theory}

\begin{figure}[H]
	\centering
	\includegraphics[width=0.99\textwidth]{/home/karl/doc/subj/att/fys4150/project2/resultsKeep/oneElectronRelativeErrorEigenvalues.pdf}
	\caption{Maximum relative error of the three smallest eigenvalues, one electron.}
	\label{1}
\end{figure}

\begin{minipage}{.49\textwidth} %\noindent
	\begin{figure}[H]
		\centering
		\includegraphics[width=0.99\textwidth]{/home/karl/doc/subj/att/fys4150/project2/resultsKeep/oneElectronOmega10RhoMaxComparison.pdf}
		\caption{Minimum eigenvalue. 1 electron.}
		\label{1}
	\end{figure}
\end{minipage}\hfill
\begin{minipage}{.49\textwidth}
	\begin{figure}[H]
		\centering
		\includegraphics[width=0.99\textwidth]{/home/karl/doc/subj/att/fys4150/project2/resultsKeep/TwoElectronNoCoulombOmega10RhoMaxComparison.pdf}
		\caption{Minimum eigenvalue. 2 electrons no Coulomb.}
		\label{1}
	\end{figure}
\end{minipage}\hfill

\begin{figure}[H]
	\centering
	\includegraphics[width=0.99\textwidth]{/home/karl/doc/subj/att/fys4150/project2/resultsKeep/TwoElectronCoulombOmega025RhoMaxComparison.pdf}
	\caption{Relative error smallest eigenvalue.}
	\label{1}
\end{figure}

\begin{minipage}{.49\textwidth} %\noindent
	\begin{figure}[H]
		\centering
		\includegraphics[width=0.99\textwidth]{/home/karl/doc/subj/att/fys4150/project2/resultsKeep/TwoElectronNoCoulombOmegaComparison.pdf}
		\caption{Minimum eigenvalue. 2 electron. No Coulomb.}
		\label{1}
	\end{figure}
\end{minipage}\hfill
\begin{minipage}{.49\textwidth}
	\begin{figure}[H]
		\centering
		\includegraphics[width=0.99\textwidth]{/home/karl/doc/subj/att/fys4150/project2/resultsKeep/TwoElectronCoulombOmegaComparison.pdf}
		\caption{Minimum eigenvalue. 2 electrons  Coulomb.}
		\label{1}
	\end{figure}
\end{minipage}\hfill

\section{Conclusions}


\section{Feedback}
\subsection{Project 1}
This project has been extremely educational. We learned about about c++, especially pointers and dynamic memory allocoation. Also which for us was a well forgotten subject, we learned about dangerous of numerical round-off errors. \\

We feel the size of the project is large, much larger than typical assignments in other courses. However, the quality and quantity of the teaching without a doubt made the workload managable. The detailed lectures, combined with the fast and good respones on Piazze helped a lot!\\

We think the project could have gone even smoother, if we on the 2nd lab-session had learned basic branching in Github. We used a considerable amount of time finding out of this.\\

All in all, two thumbs up!

\subsection{Project 2}

\pagebreak
\begin{thebibliography}{9}
	\bibitem{MHJ} 
	Hjorth-Jensen, M.(2015)
	\textit{Computational physics. Lectures fall 2015}. 
	\url{https://github.com/CompPhysics/ComputationalPhysics/tree/master/doc/Lectures}
	
	\bibitem{watkins} 
	Watkins, D.S.(2002)
	\textit{Fundamentals of matrix computations. 2nd edition.}
	
	\bibitem{kiusalaas} 
	Kiusalaas, J.(2013)
	\textit{Numerical Methods in Engineering with Python 3. 3rd edition.}

\end{thebibliography}


\end{document}
